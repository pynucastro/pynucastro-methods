%\documentclass[12pt, preprint]{aastex}
\documentclass[preprint,times,tighten]{aastex63}

\usepackage{epsf,color,amsmath}
\usepackage{multirow}
\usepackage{cancel}

\newcommand{\sfrac}[2]{\mathchoice%
  {\kern0em\raise.5ex\hbox{\the\scriptfont0 #1}\kern-.15em/
    \kern-.15em\lower.25ex\hbox{\the\scriptfont0 #2}}
  {\kern0em\raise.5ex\hbox{\the\scriptfont0 #1}\kern-.15em/
    \kern-.15em\lower.25ex\hbox{\the\scriptfont0 #2}}
  {\kern0em\raise.5ex\hbox{\the\scriptscriptfont0 #1}\kern-.2em/
    \kern-.15em\lower.25ex\hbox{\the\scriptscriptfont0 #2}} {#1\!/#2}}

\newcommand{\myhalf}{\sfrac{1}{2}}
\newcommand{\nph}{{n+\myhalf}}
\newcommand{\nmh}{{n-\myhalf}}

\newcommand{\inp}{\mathrm{in}}
\newcommand{\outp}{\mathrm{out}}

% boldsymbol means bold italic
\newcommand{\eb}{{\bf{e}}}
\newcommand{\Ub}{{\bf{U}}}
\newcommand{\xb}{{\bf{x}}}
\newcommand{\kb}{{\bf{k}}}
\newcommand{\Vb}{{\bf{V}}_n}
\newcommand{\Vbhat}{{\bf{\widehat{V}}}_n}
\newcommand{\Omegab}{{\bf{\Omega}}}
\newcommand{\gb}{{\bf{g}}}
\newcommand{\rb}{{\bf{r}}}

\newcommand{\pb}{p_\mathrm{base}}
\newcommand{\epsdot}{\dot{\epsilon}}
\newcommand{\qburn}{q_\mathrm{burn}}
\newcommand{\rt}{\tilde{r}_0}


\newcommand{\nablab}{\mathbf{\nabla}}
\newcommand{\dt}{\Delta\ t}

\newcommand{\omegadot}{\dot{\omega}}

\newcommand{\Hext}{H_{\rm ext}}
\newcommand{\Hnuc}{H_{\rm nuc}}
\newcommand{\kth}{k_{\rm th}}

\newcommand{\Gammaonebar}{\overline{\Gamma}_1}
\newcommand{\Sbar}{\overline{S}}

\newcommand{\etarho}{\eta_\rho}
\newcommand{\etarhoh}{\eta_{\rho~h}}

\newcommand{\Ubt}{\widetilde{\Ub}}
\newcommand{\wt}{\widetilde{w}}

\newcommand{\He}{$^4$He}
\newcommand{\C}{$^{12}$C}
\newcommand{\Fe}{$^{56}$Fe}

\newcommand{\isot}[2]{$^{#2}\mathrm{#1}$}
\newcommand{\isotm}[2]{{}^{#2}\mathrm{#1}}

\newcommand{\maestro}{{\sf MAESTRO}}
\newcommand{\castro}{{\sf Castro}}
\newcommand{\amrex}{{\sf AMReX}}
\newcommand{\pynucastro}{{\sf pynucastro}}

\newcommand{\avg}[1]{\overline{#1}}
\newcommand{\avgtwod}[1]{\langle~#1 \rangle}
\newcommand{\rms}[2]{\left(\delta#1\right)_{r_{#2}}}
\newcommand{\mymax}[1]{\left(#1\right)_{\rm max}}

\newcommand{\Tb}{\ensuremath{T_\mathrm{base}}}
\newcommand{\gcc}{\mathrm{g~cm^{-3} }}
\newcommand{\cms}{\mathrm{cm~s^{-1} }}

\newcommand{\half}{\frac{1}{2}}

\setlength{\marginparwidth}{0.5in}

\newcommand{\MarginPar}[1]{
    \marginpar{\vskip-\baselineskip%
               \raggedright%
               \tiny\sffamily%
               {\color{red}\hrule%
               \smallskip%
               #1\par%
               \smallskip%
               \hrule}}%
}

\newcommand{\AssignTo}[1]{
    \marginpar{\vskip-\baselineskip%
               \raggedright%
               \tiny\sffamily%
               {\color{blue}\hrule%
               \smallskip%
               #1\par%
               \smallskip%
               \hrule}}%
}

\begin{document}
%======================================================================
% Title
%======================================================================
\title{Pynucastro Methods and Verification}

\shorttitle{Pynucastro}
\shortauthors{Pynucastro Development Team}

\author[0000-0000-0000-0000]{Pynucastro Development Team}
\affiliation{Planet Earth}

\correspondingauthor{Pynucastro Development Team}
\email{weshouldprobablyhaveanemail@somewhere}


%======================================================================
% Abstract and Keywords
%======================================================================
\begin{abstract}
We present pynucastro, a software tool for interacting with nuclear
reaction rate databases for pedagogical and research purposes. We
also present verification tests for pynucastro as well as code
comparisons with other community codes for integrating general
reaction networks.
\end{abstract}

\keywords{X-ray bursts (1814), Nucleosynthesis (1131), Open source software (1866), Computational methods (1965)}

%======================================================================
% Introduction
%======================================================================
\section{Introduction}\label{Sec:Introduction}

\begin{itemize}
    \item Outline existing need for a code like this
    \item Review previous community codes for reaction networks
    \item Describe what we are doing differently (vis, code generation, ...)
    \item Introduce the verification and code comparison problems we will talk about later
    \item Discuss the big picture: what data we turn into what ODEs for what other codes (Castro, MAESTROeX)
    \item Sources of data for pynucastro (Reaclib, tabulated weak rates, CODATA, Atomic Mass Evaluation, ...)
\end{itemize}

%======================================================================
% Software Design
%======================================================================
\section{Software Design}\label{Sec:Design}

\begin{itemize}
    \item Overview of the structure of the python code comprising pynucastro 
    \item Present visualization capabilities with Jupyter notebooks 
    \item Discuss ODE construction with Sympy 
    \item Discuss code template processing system for code generation
    \item Discuss different code outputs (python networks, StarKiller Microphysics networks)
    \item Discuss CI approach and what parts of pynucastro we test
\end{itemize}

%======================================================================
% Verification 
%======================================================================
\section{Simple Verification Tests}\label{Sec:Verification}

\begin{itemize}
    \item Find a way to verify sub-components of pynucastro
    \item Present simple tests where we could get the answer analytically?
    \item At least one test for each Reaclib chapter, plus tabulated rates?
\end{itemize}

%======================================================================
% Code Comparisons
%======================================================================
\section{Code Comparisons}\label{Sec:Comparisons}

\begin{itemize}
    \item Describe other codes for the comparison (XNet, Skynet, MESA?)
    \item Outline similarities and differences in the microphysics and integration strategy for the codes
    \item Outline real problems we will compare with other codes using a range of networks
    \item XRBs? (one zone burns, thermodynamic trajectories from multi-D flames)
    \item Discuss performance?
    \item Discuss ease of use/python vs other APIs?
\end{itemize}

%======================================================================
% Conclusions/Outlook
%======================================================================
\section{Conclusions and Outlook}\label{Sec:Conclusions}

Conclude something, invite people to create issues/submit PRs.

%======================================================================
% References
%======================================================================

\bibliographystyle{aasjournal}
\bibliography{ws}


\end{document}
